\documentclass{article}

\usepackage[T1]{fontenc}
\usepackage{amsmath}
\usepackage{amssymb}
\usepackage{mathtools}
\usepackage{algpseudocode}

\DeclarePairedDelimiter{\norm}{\lVert}{\rVert}

\title{Sprawozdanie \\Metody numeryczne 2 \\\textbf{Temat 1, Zadanie nr 6}}
\date{10/15/2018}
\author{Mateusz Śliwakowski}

\begin{document}
  \pagenumbering{gobble}
  \maketitle
  \newpage
  \pagenumbering{arabic}

\section{Treść zadania}
\paragraph{}
 Metoda iteracji prostej SOR dla układu \(Ax = b\) z macierzą rozrzedzoną.
\section{Opis metody}

\subsection {Wprowadzenie - metody iteracyjne}
\paragraph{}
Służą do rozwiązywania układów równań liniowych
\[
\left\{ 
\begin{array}{c}
a_{11}x_1+a_{12}x_2+\dots+a_{1n}x_n=b_1 \\ 
a_{21}x_1+a_{22}x_2+\dots+a_{2n}x_n=b_2 \\ 
\vdots\\
a_{n1}x_1+a_{n2}x_2+\dots+a_{nn}x_n=b_n \\ 
\end{array}
\right. 
\]
Będziemy je zapisywać w postaci \(Ax = b\) zakładając: \\
\[A=\begin{bmatrix}
  a_{11} & a_{21} & \dots & a_{n1}\\
  a_{21} & a_{22} & \dots & a_{n2}\\
 \vdots & \vdots & \ddots & \vdots \\
  a_{n1} & a_{n2} & \dots & a_{nn}
\end{bmatrix},\hspace{2pt}
 x=\begin{bmatrix}
 x_1\\
 x_2\\
 \vdots\\
 x_n
\end{bmatrix}, \hspace{2pt}
 b=\begin{bmatrix}
 b_1\\
 b_2\\
 \vdots\\
 b_n
\end{bmatrix}\]
\paragraph{}
W metodzie iteracyjnej tworzony jest ciąg kolejnych przybliżeń 
\((x^{(1)}, x^{(2)},\dots)\)
taki, że o ile metoda jest zbieżna \(
\norm{x^{(k)}-x}\xrightarrow{k\rightarrow\infty}x
\).
\paragraph{}
Wiele metod iteracyjnych (w tym metoda SOR) ma postać \\ \(x^{(k+1)}=Bx^{(k)}+C\), gdzie \(B\in \mathbb{R}^{n\times n}\), a \(C\in\mathbb{R}^{n}\). Postać $B$ i $C$ definuje konkretną metodę.
\subsection {Metoda SOR (Successive Over-Relaxation)}
\paragraph{}
Zdefiniujmy macierze:
\[
L = \begin{bmatrix}
  0 & 0 & 0 & \dots &0\\
  a_{21} &0 & 0 & \dots & 0\\
  a_{31} & a_{32} &0  & \dots & 0\\
 \vdots & \vdots &\vdots & \ddots & \vdots \\
  a_{n1} & a_{n2} &a_{n3} & \dots 0
\end{bmatrix},\]
\[
D = \begin{bmatrix}
  a_{11} & 0 & 0 & \dots &0\\
  0 & a_{22} & 0 & \dots & 0\\
  0 & 0  & a_{33} & \dots & 0\\
 \vdots & \vdots &\vdots & \ddots & \vdots \\
  0 & 0 & 0 & \dots & a_{nn}
\end{bmatrix},\]
\[
U = \begin{bmatrix}
  0 & a_{12} & \dots & a_{1n-1} & a_{1n} \\
  0 & 0 & \dots &  a_{2n-1} & a_{2n}\\
 \vdots & \vdots &\ddots & \vdots & \vdots \\
  0 & 0  &\dots &  0 & a_{n-1n}\\
  0 & 0 & 0 & 0 & 0
\end{bmatrix}
 \]
\paragraph{}
Wprowadźmy parametr relaksacji $\omega \in \mathbb{R}$.
Wzór macierzowy metody SOR ma postać:
\[
x^{(k+1)}=\underbrace{(D+\omega L)^{-1}((1-\omega)D-\omega U)}_{B_{SOR}}x^k + \underbrace{(D+ \omega L)^{-1}b\omega}_{C_{SOR}}
\]
Zapis niemacierzowy:
\begin{algorithmic}
\Repeat {
\For {$i=1,\dots,n$}
\State
$x_i^{k+1}=(1-\omega)x_i^{(k)}+\omega(b-
	\sum_{j=1}^{i-1}a_{ij}x_j^{(k+1)}-
	\sum_{j=i+1}^{n}a_{ij}x_j^{(k)}
	)/a_{ii}$
\EndFor
}
\Until{<uzyskamy zbieżność>}
\end{algorithmic}


\section{Warunki i założenia}

\section{Implementacja metody}

\section{Przykłady i wnioski}

\end{document}