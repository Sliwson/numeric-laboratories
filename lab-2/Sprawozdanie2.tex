\documentclass{article}

\usepackage[T1]{fontenc}
\usepackage{amsmath}
\usepackage{amssymb}
\usepackage{mathtools}
\usepackage{algpseudocode}
\usepackage[inline]{enumitem}
\usepackage{listings}
\usepackage{matlab-prettifier}
\usepackage{pgfplots}
\usepackage[Polish]{babel}

\DeclarePairedDelimiter{\norm}{\lVert}{\rVert}

\title{Sprawozdanie \\Metody numeryczne 2 \\\textbf{Temat 2, Zadanie nr 12}}
\date{28/10/2018}
\author{Mateusz Śliwakowski, F4}

\begin{document}
  \pagenumbering{gobble}
  \maketitle
 	  \newpage
  \pagenumbering{arabic}

\section{Treść zadania}
\paragraph{}
Interpolacja funkcjami kwadratowymi na obszarze $D: \vert x \vert + \vert y \vert \leq 1$ podzielonym na $4n^2$ trójkątów przystających. Tablicowanie funkcji, przybliżenia i błędu w środkach ciężkości trójkątów. Obliczenie błędu maksymalnego w tych punktach.
\section{Opis metody}
\subsection{Podział obszaru}
\paragraph{}
Na początku zobaczmy jak wygląda obszar $D: \vert x \vert + \vert y \vert \leq 1$.\\\\
\begin{tikzpicture}
	\begin{axis}
	\addplot+ [fill] coordinates {
	(0,1) (1,0) (0,-1) (-1,0)
	}
	--cycle
;
\end{axis}
\end{tikzpicture}\\
Podzielimy go na 4 trójkąty prostokątne:\\\\
\begin{tikzpicture}
	\begin{axis}
		\addplot [color=black] coordinates {
		(0,1) (1,0) (0,-1) (-1,0)
		}
		--cycle;
		\addplot [color=black] coordinates {
		(0,1) (0,-1)
		};
		\addplot [color=black] coordinates {
		(-1,0) (1,0)
		};
	\end{axis}
\end{tikzpicture}\\
Każdy z trójkątów podzielimy na $n^2$ trójkątów przystających w następujący sposób:
\begin{enumerate}
\item Przyprostokątne dzielimy na n odcinków równej długości.
\item Tworzymy odcinki łączące punkty podziału z przeciwprostokątną równoległe do osi odpowiednio $OX$ oraz $OY$.
\item W każdym z powstałych kwadratów prowadzimy jedną przekątną.
\end{enumerate}
Tym sposobem w trójkącie otrzymujemy $1 + 3 + \dots + 2n-1 = n^2$ trójkątów, zatem podział jest poprawny.
\subsection{Interpolacja funkcjami kwadratowymi}
\paragraph{}
Niech $f$ - interpolowana funkcja dwóch zmiennych. Będziemy przybliżać tę funkcję za pomocą funkcji kwadratowej postaci $w(x,y) = a_0 + a_1x + a_2y + a_3xy + a_4x^2 + a_5y^2$, gdzie dla $i=0,\dots,5$  $a_i \in \mathbb{R}$.
\paragraph{}
Aby wyznaczyć współczynniki funkcji interpolacyjnej $w$ musimy skorzystać z wartości funkcji $f$ w 6 punktach. Weźmy zatem wierzchołki trójkąta (oznaczmy je $P_0, P_1, P_2$) oraz środki boków (oznaczmy $P_{01}, P_{12}, P_{20}$). Przyjmijmy ponadto, że $P_i := (x_i, y_i)$. Utwórzmy zatem układ równań:
$$
\left\{ 
\begin{array}{c}
w(x_0,y_0) = f(x_0, y_0) \\ 
w(x_1,y_1) = f(x_1, y_1) \\ 
w(x_2,y_2) = f(x_2, y_2) \\ 
w(x_{01},y_{01}) = f(x_{01}, y_{01}) \\
w(x_{12},y_{12}) = f(x_{12}, y_{12}) \\
w(x_{20},y_{20}) = f(x_{20}, y_{20}) \\ 
\end{array}
\right.
$$
Rozwiązując powyższy URL otrzymamy współczynniki szukanej funkcji $w$.
\section{Implementacja metody}
\paragraph{}
\begin{lstlisting}[style=Matlab-editor]
function [B, err] =  squareInterpolation(fun, n)
\end{lstlisting}
\vspace{4pt}
Parametry wejściowe:
\begin{itemize}
\item $fun$ - Uchwyt do interpolowanej funkcji dwóch zmiennych,
\item $n$ - liczba naturalna, parametr zadania.
\end{itemize}
Parametry wyjściowe:
\begin{itemize}
\item $B$ - tablica zawierająca: współrzędne x oraz y środków ciężkości trójkątów, wartości funkcji interpolowanej w tych punktach, wartości funkcji interpolacyjnej w tych punktach, błąd przybliżenia.
\item $err$ - maksymalny błąd interpolacji.
\end{itemize}
\paragraph{}
Na początku musimy znaleźć wierzchołki trójkątów wymaganych podczas interpolacji. W tym celu za pomocą funkcji $meshgrid$ dzielimy obszar na siatkę równoodległych punktów na obszarze $[-1,1]\times[-1,1]$. Następnie wyznaczamy podział lewego górnego trójkąta i zapisujemy go w wektorze pomocniczym. Podział ten odbijamy wzdłuż osi $OY$, a następnie $OX$ otrzymując wszystkie wymagane punkty.
\paragraph{}
Potem następuje właściwa interpolacja. Dla każdego z trójkątów:
\begin{itemize}
\item Zapisujemy do tablicy wynikowej współrzędne środka ciężkości danego trójkąta.
\item Zapisujemy wartość funkcji interpolowanej w środku ciężkości.
\item Wyznaczamy współczynniki funkcji interpolacyjnej.
\item Tablicujemy wartość funkcji interpolacyjnej oraz błąd interpolacji.
\end{itemize}
Na koniec wyznaczamy maksymalny błąd.
\paragraph{}
Dla uproszczenia kodu, użyta została prosta funkcja $initializeAFromT$, która konwertuje współrzędne iteracyjne (z zakresu $[1,n]$) na współrzędne na płaszczyźnie.
\section{Przykłady i wnioski}
\subsection{Przykłady}
Wszystkie przedstawione przykłady zdefiniowane są w skrypcie $testInterpolation.m$. Najpierw przetestujemy funkcję interpolacyjną pod kątem maksymalnego błędu przybliżenia. W ramkach przedstawiona jest część wyjścia Matlaba po uruchomieniu skryptu $testInterpolation$.\\
\begin{itemize}
\item
Rozpoczniemy od przykładu, gdzie nasz program powinien działać praktycznie bezbłędnie, czyli wywołamy funkcję $squareInterpolation$ dla funkcji kwadratowej dwóch zmiennych.\\\\
\noindent\fbox{
    \parbox{\textwidth}{
    \vspace{3pt}
    ===============TEST CASE===============\\
    $@(x,y)\hspace{5pt}2-23*x+30*y+4*x*y-2.5*x*x+9*y*y$\\

	n = 1: maxError: 0\\
	n = 10: maxError: 1.4211e-14\\
	n = 100: maxError: 1.7764e-14\\
	n = 200: maxError: 1.7764e-14
	}
}\\\\
Zgodnie z oczekiwaniami otrzymujemy dokładność porównywalną z dokładnością maszynową.
\item
Zobaczmy jak program zachowa się dla funkcji złożonej z funkcji trygonometrycznych.\\\\
\noindent\fbox{
    \parbox{\textwidth}{
    \vspace{3pt}
	===============TEST CASE===============\\
    $@(x,y)\hspace{5pt}sin(x)+3*cos(y)-4*sin(x*y)+8*cos(x*x*y)$\\

	n = 1: maxError: 0.039506\\
	n = 10: maxError: 0.00026801\\
	n = 100: maxError: 3.1845e-07\\
	n = 200: maxError: 4.0173e-08
	}
}\\\\
Program zachowuje się poprawnie - wraz ze wzrostem liczby trójkątów, których używamy do interpolacji rośnie dokładność.
\item
Dla wielomianu wyższego stopnia:\\\\
\noindent\fbox{
    \parbox{\textwidth}{
    \vspace{3pt}
	===============TEST CASE===============\\
    $@(x,y)\hspace{5pt}@4*x^9+3*x^3+8*x-4*y^8+3*y^2;$\\

	n = 1: maxError: 0.97998\\
	n = 10: maxError: 0.00906\\
	n = 100: maxError: 1.2157e-05\\
	n = 200: maxError: 1.5444e-06
	}
}\\\\
Również osiągamy dobrą dokładność dla wysokich n, lecz już nie tak wysoką jak dla funkcji trygonometrycznej.
\item
Na koniec sprawdźmy jak program zadziała dla funkcji, która w zadanym przedziale ma punkty, dla których nie jest określona.\\\\
\noindent\fbox{
    \parbox{\textwidth}{
    \vspace{3pt}
	===============TEST CASE===============\\
	$@(x,y)\hspace{5pt}tan(2*x+3*y)$\\

	n = 1: maxError: 17.2849\\
	n = 10: maxError: 216.6608\\
	n = 100: maxError: 599.0144\\
	n = 200: maxError: 770.5785
    }
}\\\\
W tym przypadku nie jesteśmy w stanie przeprowadzić poprawnej interpolacji - dla wszystkich wartości n maksymalny błąd osiąga wysokie wartości. Dzieje się tak dlatego, że w otoczeniu punktów, gdzie funkcja tangens nie jest określona jej pochodna osiąga bardzo wysokie wartości - nie jest zatem możliwa dokładna interpolacja.
\end{itemize}
\newpage
\subsection{Doświadczalne wyznaczenie współczynnika zbieżności}
Sprawdźmy teraz jaki jest współczynnik zbieżności naszej metody. W każdej iteracji będziemy zwiększać parametr n dwa razy i wyświetlać iloraz kolejnych błędów.
\\\\
\noindent\fbox{
    \parbox{\textwidth}{
    \vspace{3pt}
	5.7240\\
    7.6428\\
    7.3948\\
    7.7738\\
    7.9081\\
    7.9693\\
    7.9837\\
    7.9916
    }
}\\\\
Jak widać $n\rightarrow8$ zatem współczynnik zbieżności może wynosić 3 (bo $2^3=8$).
\end{document}