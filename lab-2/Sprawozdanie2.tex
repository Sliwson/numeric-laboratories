\documentclass{article}

\usepackage[T1]{fontenc}
\usepackage{amsmath}
\usepackage{amssymb}
\usepackage{mathtools}
\usepackage{algpseudocode}
\usepackage[inline]{enumitem}
\usepackage{listings}
\usepackage{matlab-prettifier}
\usepackage{pgfplots}
\usepackage[Polish]{babel}

\DeclarePairedDelimiter{\norm}{\lVert}{\rVert}

\title{Sprawozdanie \\Metody numeryczne 2 \\\textbf{Temat 2, Zadanie nr 12}}
\date{28/10/2018}
\author{Mateusz Śliwakowski, F4}

\begin{document}
  \pagenumbering{gobble}
  \maketitle
	  \newpage
  \pagenumbering{arabic}

\section{Treść zadania}
\paragraph{}
Interpolacja funkcjami kwadratowymi na obszarze $D: \vert x \vert + \vert y \vert \leq 1$ podzielonym na $4n^2$ trójkątów przystających. Tablicowanie funkcji, przybliżenia i błędu w środkach ciężkości trójkątów. Obliczenie błędu maksymalnego w tych punktach.
\section{Opis metody}
\subsection{Podział obszaru}
\paragraph{}
Na początku zobaczmy jak wygląda obszar $D: \vert x \vert + \vert y \vert \leq 1$.\\\\
\begin{tikzpicture}
	\begin{axis}
	\addplot+ [fill] coordinates {
	(0,1) (1,0) (0,-1) (-1,0)
	}
	--cycle
;
\end{axis}
\end{tikzpicture}\\
Podzielimy go na 4 trójkąty prostokątne:\\\\
\begin{tikzpicture}
	\begin{axis}
		\addplot [color=black] coordinates {
		(0,1) (1,0) (0,-1) (-1,0)
		}
		--cycle;
		\addplot [color=black] coordinates {
		(0,1) (0,-1)
		};
		\addplot [color=black] coordinates {
		(-1,0) (1,0)
		};
	\end{axis}
\end{tikzpicture}\\
Lewy górny trójkąt podzielimy na $n^2$ trójkątów przystających w następujący sposób:
\begin{enumerate}
\item Przyprostokątne dzielimy na n odcinków równej długości.
\item Tworzymy odcinki łączące punkty podziału z przeciwprostokątną równoległe do osi odpowiednio $OX$ oraz $OY$.
\item W każdym z powstałych kwadratów prowadzimy jedną przekątną.
\end{enumerate}
Tym sposobem w trójkącie otrzymujemy $1 + 3 + \dots + 2n-1 = n^2$ trójkątów, zatem podział jest poprawny.
\subsection{Interpolacja funkcjami kwadratowymi}
\paragraph{}
Niech $f$ - interpolowana funkcja dwóch zmiennych. Będziemy przybliżać ją funkcją kwadratową w postaci $w(x,y) = a_0 + a_1x + a_2y + a_3xy + a_4x^2 + a_5y^2$, gdzie dla $i=0,\dots,5$  $a_i \in \mathbb{R}$.
\paragraph{}

\section{Warunki i założenia}
\section{Implementacja metody}
\section{Przykłady i wnioski}

\end{document}