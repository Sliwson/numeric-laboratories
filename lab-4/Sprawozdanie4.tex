\documentclass{article}

\usepackage[T1]{fontenc}
\usepackage{amsmath}
\usepackage{amssymb}
\usepackage{mathtools}
\usepackage{algpseudocode}
\usepackage[inline]{enumitem}
\usepackage{listings}
\usepackage{matlab-prettifier}
\usepackage{pgfplots}
\usepackage[Polish]{babel}

\DeclarePairedDelimiter{\norm}{\lVert}{\rVert}

\title{Sprawozdanie \\Metody numeryczne 2 \\\textbf{Temat 4, Zadanie nr 12}}
\date{4/12/2018}
\author{Mateusz Śliwakowski, F4}

\begin{document}
  \pagenumbering{gobble}
  \maketitle
 	  \newpage
  \pagenumbering{arabic}

\section{Treść zadania}
\paragraph{}
Wzory empiryczne. Baza: $1,x,y,x^2y^2$. Tablicowanie błędów w punktach pomiarowych oraz obliczanie błędu średniokwadratowego. Punkty pomiarowe wybieramy z prostokąta $[a,b]\times[c,d]$.
\section{Opis metody}
\end{document}