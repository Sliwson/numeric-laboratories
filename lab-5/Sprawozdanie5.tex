\documentclass{article}

\usepackage[T1]{fontenc}
\usepackage{amsmath}
\usepackage{amssymb}
\usepackage{mathtools}
\usepackage{algpseudocode}
\usepackage[inline]{enumitem}
\usepackage{listings}
\usepackage[framed]{matlab-prettifier}
\usepackage{pgfplots}
\usepackage[Polish]{babel}
\usepackage{float}

\DeclarePairedDelimiter{\norm}{\lVert}{\rVert}

\title{Sprawozdanie \\Metody numeryczne 2 \\\textbf{Temat 5, Zadanie nr 6}}
\date{28/12/2018}
\author{Mateusz Śliwakowski, F4}

\begin{document}
  \pagenumbering{gobble}
  \maketitle
 	  \newpage
  \pagenumbering{arabic}

\section{Treść zadania}
\paragraph{}
Odwrotna metoda potęgowa z normowaniem dla macierzy trójdiagonalnej. Poszukiwanie wartości własnej macierzy A leżącej najbliżej podanej wartości własnej $\lambda^*$. Układ równań z macierzą $A - \lambda^*I$ należy rozwiązać metodą dla macierzy trójdiagonalnych.
\section{Opis metody}
\subsection{Odwrotna metoda potęgowa}
\paragraph{}
Metoda potęgowa służy do

\end{document}