\documentclass{article}

\usepackage[T1]{fontenc}
\usepackage{amsmath}
\usepackage{amssymb}
\usepackage{mathtools}
\usepackage{algpseudocode}
\usepackage[inline]{enumitem}
\usepackage{listings}
\usepackage[framed]{matlab-prettifier}
\usepackage{pgfplots}
\usepackage[Polish]{babel}
\usepackage{float}

\DeclarePairedDelimiter{\norm}{\lVert}{\rVert}

\title{Sprawozdanie \\Metody numeryczne 2 \\\textbf{Temat 5, Zadanie nr 6}}
\date{28/12/2018}
\author{Mateusz Śliwakowski, F4}

\begin{document}
  \pagenumbering{gobble}
  \maketitle
 	  \newpage
  \pagenumbering{arabic}

\section{Treść zadania}
\paragraph{}
Odwrotna metoda potęgowa z normowaniem dla macierzy trójdiagonalnej. Poszukiwanie wartości własnej macierzy A leżącej najbliżej podanej wartości własnej $\lambda^*$. Układ równań z macierzą $A - \lambda^*I$ należy rozwiązać metodą dla macierzy trójdiagonalnych.
\section{Opis metody}
\subsection{Odwrotna metoda potęgowa}
\paragraph{}
Odwrotna metoda potęgowa korzysta z następującego twierdzenia:\\
\textit{Jeśli $\lambda$ jest wartością własną nieosobliwej macierzy $A$ to $\lambda^-1$ jest wartością własną macierzy $A^-1$.}
Polega ona na zastosowaniu metody potęgowej do macierzy $A^-1$.
\paragraph{}
Jeśli dobierzemy $\lambda^*$ takie, że $det(A-\lambda^{*}I)\neq0$ to $$\lambda \in \sigma(A) \Leftrightarrow \lambda \neq \lambda^{*} \wedge \frac{1}{\lambda-\lambda^{*}} \in \sigma(A-\lambda^* I)^{-1}$$ 
\paragraph{}
Zatem aby wyznaczyć wartość własną A, najbliższą danej $\lambda^*$, należy zastosować metodę potęgową do macierzy $(A-\lambda^*I)^{-1}$.\\
Ze względów wydajnościowych nie należy obliczać macierzy $(A-\lambda^*I)^{-1}$ (jest to kosztowna operacja). Pojedynczy krok metody potęgowej wykonujemy rozwiązując układ równań liniowych $(A - \lambda^*I)x^{(k)} = x^{(k-1)}$. Szczegółowy opis metody rozwiązywania tego układu zostanie opisany w następnej sekcji sprawozdania.
\subsection{Rozwiązywanie układu równań z macierzą trójdiagonalną}
\paragraph{}
Mając macierz trójdiagonalną, nie jest optymalnym przechowywać ją całą w pamięci - lepszym rozwiązaniem jest zapisywanie trzech wektorów zawierających elementy kolejnych diagonali. Wygodnymi sposobami dla rozwiązywania takich równań są m. in. metody iteracyjne, czy metoda Thomasa. Jednak na potrzeby zadania zdecydowałem się na zaimplementowanie metody GEPP dla tego szczególnego przypadku, ponieważ jest ona bardziej niezawodna niż wyżej podane.
\paragraph{}
Algorytm GEPP:\\
Załóżmy, że mamy daną macierz $A\in\mathbb{R}^{n\times n}$ - trójdiagonalna. Dla $k=1:n-1$:
\begin{enumerate}
\item Sprawdź czy $|A[k+1,k]| > |A[k,k]|$.
\item Jeśli tak to zamień wiersze $k+1$ i $k$-ty (w naszym przypadku zamieniamy elementy w wektorach).
\item Od wiersza $k+1$ odejmij $w_{k}*A[k+1,k]/A[k+1,k]$ (również posługujemy się operacjami na wektorach).
\end{enumerate}
Następnie obliczamy elementy wektora wynikowego korzystając z prostego algorytmu dla macierzy trójkątnej górnej.
\section{Implementacja metody}

\end{document}