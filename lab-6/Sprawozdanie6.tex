\documentclass{article}

\usepackage[T1]{fontenc}
\usepackage{amsmath}
\usepackage{amssymb}
\usepackage{mathtools}
\usepackage{algpseudocode}
\usepackage[inline]{enumitem}
\usepackage{listings}
\usepackage[framed]{matlab-prettifier}
\usepackage{pgfplots}
\usepackage[Polish]{babel}
\usepackage{float}

\DeclarePairedDelimiter{\norm}{\lVert}{\rVert}

\title{Sprawozdanie \\Metody numeryczne 2 \\\textbf{Temat 6, Zadanie nr 5}}
\date{08/01/2019}
\author{Mateusz Śliwakowski, F4}

\begin{document}
  \pagenumbering{gobble}
  \maketitle
 	  \newpage
  \pagenumbering{arabic}

\section{Treść zadania}
\paragraph{}
Wahadło matematyczne. Zastosować wybraną metodę Rungego-Kutty rzędu 2-go dla układu dwóch równań. Patrz "Ćwiczenia laboratoryjne z metod numerycznych" pod redakcją J. Wąsowskiego, OWPW, podrozdział 7.6.
\section{Opis metody}
\end{document}