\documentclass{article}

\usepackage[T1]{fontenc}
\usepackage{amsmath}
\usepackage{amssymb}
\usepackage{mathtools}
\usepackage{algpseudocode}
\usepackage[inline]{enumitem}
\usepackage{listings}
\usepackage{matlab-prettifier}
\usepackage{pgfplots}
\usepackage[Polish]{babel}

\DeclarePairedDelimiter{\norm}{\lVert}{\rVert}

\title{Sprawozdanie \\Metody numeryczne 2 \\\textbf{Temat 3, Zadanie nr 1}}
\date{20/11/2018}
\author{Mateusz Śliwakowski, F4}

\begin{document}
  \pagenumbering{gobble}
  \maketitle
 	  \newpage
  \pagenumbering{arabic}

\section{Treść zadania}
\paragraph{}
Obliczanie całek $\iint_D f(x,y) \,dx\,dy$, gdzie $D = [a,b]\times[c,d]$ złożonymi kwadraturami trapezów ze względu na każdą zmienną.
\section{Opis metody}
\paragraph{}
Zaczniemy od rozpisania ogólnego przypadku liczenia całki na prostokącie, a następnie zastosujemy kwadraturę trapezów ze względu na każdą zmienną.
\paragraph{}
Mamy dane: $f(x,y)$, $D = [a,b]\times[c,d]$, gdzie $a,b,c,d\in\mathbb{R}$ oraz $n,m\in\mathbb{R}$ - ilość podziałów wzdłuż osi odpowiednio $OX$, $OY$. Zdefiniujmy podziały:
$$x_i=a+h_1i\text{, gdzie }h_1=\dfrac{b-a}{n}\text{, }i=0,\dots,n$$
$$_i=a+h_2j\text{, gdzie }h_2=\dfrac{d-c}{m}\text{, }j=0,\dots,n$$
Załóżmy, że $S_1$ i $S_2$ są danymi kwadraturami dla funkcji jednej zmiennej $g$.\\
$$S_1(g) = \Sigma_{I=0}^{n}A_ig(x_i)=\int_a^bg(x)dx$$
$$S_2(g) = \Sigma_{j=0}^{M}B_jg(y_i)=\int_c^dg(y)dy$$
Zastosujemy $S_1$ do funkcji $f=f(x,y)$, ze względu na x.
$$S_1(f)(y)=\Sigma_{i=0}^nA_if(x_i,y)$$
a następnie $S_2$, ze względu na y
$$S_2(S_1(f))=\Sigma_{j=0}^mB_j(\Sigma_{i=0}^nA_if(x_i,y_j))=\Sigma_{i=0}^n\Sigma_{j=0}^m\underbrace{A_iB_j}_{C_{ij}}f(x_i,y_j)$$
Kwadratura złożona trapezów ma postać:
$$S_1(g)=\frac{h_1}{2}[g(a)+g(b)+2\Sigma_{i=1}^{n-1}g(x_i)]$$
Stosując kwadraturę trapezów ze względu na każdą ze zmiennych otrzymamy macierz współczynników:
$$\{C_{ij}\}=\{A_iB_j\}=\frac{h_1}{2}
\begin{bmatrix}
1\\
2\\
\vdots\\
2\\
1\\
\end{bmatrix}
\frac{h_2}{2}
\begin{bmatrix}
1 & 2 & \dots & 2 & 1
\end{bmatrix}=\frac{h1h_2}{4}
\begin{bmatrix}
1 & 2 & 2 & \dots & 2 & 1\\
2 & 4 & 4 & \dots & 4 & 2\\
\vdots & \vdots & \vdots & \ddots & \vdots & \vdots\\
2 & 4 & 4 & \dots & 4 & 2\\
1 & 2 & 2 & \dots & 2 & 1
\end{bmatrix}$$
\end{document}