\documentclass{article}

\usepackage[T1]{fontenc}
\usepackage{amsmath}
\usepackage{amssymb}
\usepackage{mathtools}
\usepackage{algpseudocode}
\usepackage[inline]{enumitem}
\usepackage{listings}
\usepackage{matlab-prettifier}
\usepackage{pgfplots}
\usepackage[Polish]{babel}

\DeclarePairedDelimiter{\norm}{\lVert}{\rVert}

\title{Sprawozdanie \\Metody numeryczne 2 \\\textbf{Temat 3, Zadanie nr 1}}
\date{20/11/2018}
\author{Mateusz Śliwakowski, F4}

\begin{document}
  \pagenumbering{gobble}
  \maketitle
 	  \newpage
  \pagenumbering{arabic}

\section{Treść zadania}
\paragraph{}
Obliczanie całek $\iint_D f(x,y) \,dx\,dy$, gdzie $D = [a,b]\times[c,d]$ złożonymi kwadraturami trapezów ze względu na każdą zmienną.
\section{Opis metody}
\paragraph{}
Zaczniemy od rozpisania ogólnego przypadku liczenia całki na prostokącie, a następnie zastosujemy kwadraturę trapezów ze względu na każdą zmienną.
\paragraph{}
Mamy dane: $f(x,y)$, $D = [a,b]\times[c,d]$, gdzie $a,b,c,d\in\mathbb{R}$ oraz $n,m\in\mathbb{N}$ - ilość podziałów wzdłuż osi odpowiednio $OX$, $OY$. Zdefiniujmy podziały:
$$x_i=a+h_1i\text{, gdzie }h_1=\dfrac{b-a}{n}\text{, }i=0,\dots,n$$
$$y_j=c+h_2j\text{, gdzie }h_2=\dfrac{d-c}{m}\text{, }j=0,\dots,m$$
Załóżmy, że $S_1$ i $S_2$ są danymi kwadraturami dla funkcji jednej zmiennej $g$.\\
$$S_1(g) = \Sigma_{I=i=0}^{n}A_ig(x_i)=\int_a^bg(x)dx$$
$$S_2(g) = \Sigma_{j=0}^{m}B_jg(y_j)=\int_c^dg(y)dy$$
Zastosujemy $S_1$ do funkcji $f=f(x,y)$, ze względu na x.
$$S_1(f)(y)=\Sigma_{i=0}^nA_if(x_i,y)$$
a następnie $S_2$, ze względu na y
$$S_2(S_1(f))=\Sigma_{j=0}^mB_j(\Sigma_{i=0}^nA_if(x_i,y_j))=\Sigma_{i=0}^n\Sigma_{j=0}^m\underbrace{A_iB_j}_{C_{ij}}f(x_i,y_j)$$
Kwadratura złożona trapezów ma postać:
$$S_1(g)=\frac{h_1}{2}[g(a)+g(b)+2\Sigma_{i=1}^{n-1}g(x_i)]$$
Stosując kwadraturę trapezów ze względu na każdą ze zmiennych otrzymamy macierz współczynników:
$$\{C_{ij}\}=\{A_iB_j\}=\frac{h_1}{2}
\begin{bmatrix}
1\\
2\\
\vdots\\
2\\
1\\
\end{bmatrix}
\frac{h_2}{2}
\begin{bmatrix}
1 & 2 & \dots & 2 & 1
\end{bmatrix}=\frac{h_1h_2}{4}
\begin{bmatrix}
1 & 2 & 2 & \dots & 2 & 1\\
2 & 4 & 4 & \dots & 4 & 2\\
\vdots & \vdots & \vdots & \ddots & \vdots & \vdots\\
2 & 4 & 4 & \dots & 4 & 2\\
1 & 2 & 2 & \dots & 2 & 1
\end{bmatrix}$$
\section{Warunki i założenia}
\begin{enumerate}
\item $f(x,y)$ jest całkowalna na obszarze $D$.
\item Podany obszar $D$ jest poprawny, tj. $a<b$ oraz $c<d$.
\item Liczba podziałów jest poprawna, tj. $n >0$ oraz $m > 0$, liczby te są całkowite.
\end{enumerate}
\section{Implementacja metody}
\paragraph{}
\begin{lstlisting}[style=Matlab-editor]
function result = trapezeInterpolation(fun, a, b, c, d, n, m)
\end{lstlisting}
\vspace{4pt}
Parametry wejściowe:
\begin{itemize}
\item $fun$ - Uchwyt do funkcji dwóch zmiennych, z której ma być policzona całka,
\item $a,b,c,d$ - liczby rzeczywiste, definiujące obszar $D$,
\item $n,m$ - ilość podziałów odpowiednio wzdłuż osi $OX$ i $OY$.
\end{itemize}
Parametry wyjściowe:
\begin{itemize}
\item $result$ - zadana całka.
\end{itemize}
\paragraph{}
Na początku definiujemy szerokości pojedynczych podziałów, oraz wyznaczamy wektory punktów podziału. Obliczenie całki następuje w podwójnej pętli \textit{for}, jak w rozdziale \textit{Opis metody}. Jako że macierz $C$ nie ma skomplikowanej postaci, to poszczególne elementy dostajemy przy użyciu pomocniczej funkcji \textit{getCoefficient}, przyjmującej indeksy $x$, $y$ oraz liczbę podziałów wzdłuż osi $OX$ oraz $OY$.
\section{Przykłady i wnioski}
Testy zdefiniowane są w pliku \textit{test.m}. Prezentowane ramki będą zawierały sformatowane wyjście funkcji \textit{compareMatlab}. Funkcja ta porównuje działanie programu z obliczeniami uzyskanymi za pomocą wbudowanej funkcji Matlaba \textit{integral2} oraz z wynikiem działania pakietu symbolicznego.
\begin{enumerate}
\item Rozpocznijmy testy od funkcji stałej, która powinna dać wynik dokładny.\\\\
\noindent\fbox{
    \parbox{\textwidth}{
    \vspace{3pt}
	    $@(x,y)1$\\
		$D = [1,3]\times[1,3]$\\
		$n = 10, m = 10$\\
		$result = 4$\\
		$Matlab = 4$\\
		$Symbolic = 4$
    }
}\\\\
\item Kolejna funkcja, która powinna dać wynik dokładny to funkcja liniowa.\\\\
\noindent\fbox{
    \parbox{\textwidth}{
    \vspace{3pt}
	    $@(x,y)2x+3y$\\
		$D = [1,3]\times[1,3]$\\
		$n = 10, m = 10$\\
		$result = 40$\\
		$Matlab = 40$\\
		$Symbolic = 40$
    }
}\\\\
\item Przybliżając całkę z funkcji kwadratowej powinniśmy otrzymać pierwsze błędy.\\\\
\noindent\fbox{
    \parbox{\textwidth}{
    \vspace{3pt}
	    $@(x,y)-3.*x.*x+2.*y.*y+3.*x.*y-8$\\
		$D = [1,3]\times[1,3]$\\
		$n = 10, m = 10$\\
		$result = 1.36$\\
		$Matlab = -1.333333333$\\
		$Symbolic = -1.333333333$
    }
}\\\\
\item Sprawdźmy funkcję trygonometryczną. Tym razem podzielimy obszar na więcej części.\\\\
\noindent\fbox{
    \parbox{\textwidth}{
    \vspace{3pt}
	    $@(x,y)sin(x).*cos(y)$\\
		$D = [1,3]\times[1,3]$\\
		$n = 500, m = 500$\\
		$result = -1.071740602$\\
		$Matlab = -1.07174346$\\
		$Symbolic = -1.07174346$
    }
}\\\\
Błąd obliczeń nie jest duży, lecz podział obszaru był gesty oraz wykonanie funkcji zajmuje więcej czasu niż wykonanie \textit{integral2}.
\item Następnie zbadamy współczynnik zbieżności metody trapezów. Weźmy $f(x,y)= 2 .* x .* x .* y - 3.*cos(x.*y) + 3$ - złożenie funkcji wielomianowej i trygonometrycznej. Zaczynając od 10 będziemy zwiększać gęstość obszaru cztery razy w każdej iteracji.\\\\
\noindent\fbox{
    \parbox{\textwidth}{
    \vspace{3pt}
		$n = 10$; $error = 0.062626$\\
		$n = 20$; $error = 0.01566$9\\
		$n = 40$; $error = 0.0039181$\\
		$n = 80$; $error = 0.00097956$\\
		$n = 160$; $error = 0.00024489$\\
		$n = 320$; $error = 6.1224e-05$\\
		$n = 640$; $error = 1.5306e-05$\\
		$n = 1280$; $error = 3.8265e-06$
    }
}\\\\
\noindent\fbox{
    \parbox{\textwidth}{
    \vspace{3pt}
    	Stosunki kolejnych błędów:\\
		$3.9968$\\
		$3.9992$\\
		$3.9998$\\
		$4.0000$\\
		$4.0000$\\
		$4.0000$\\
		$4.0000$

    }
}\\\\
Za każdym razem, gdy zwiększamy gęstość obszaru cztery razy, błąd zmniejsza się cztery razy, zatem współczynnik zbieżności metody wynosi 1 - nie jest to zbyt wysoka wartość.
\item Na koniec zobaczmy jak program zachowa się dla funkcji niecałkowalnej na danym obszarze.\\\\
\noindent\fbox{
    \parbox{\textwidth}{
    \vspace{3pt}
    	$@(x,y)tan(x./y)$\\
		$D = [-1,1]\times[-1,1]$\\
		$n = 5000, m = 5000$\\
		$result = NaN$\\
		$Matlab = -0.0007325434555$\\
		Symbolic function cannot compute result
	}
}\\\\
Jak widzimy wynik działania naszego programu nie jest liczbą. Pakiet symboliczny również nie zwrócił wyniku. Co ciekawe, wynik będący liczbą otrzymaliśmy przy użyciu funkcji \textit{integral2}. Nierozważny użytkownik, bez analizy wyglądu funkcji mógłby uznać go za poprawny.
\end{enumerate}
\paragraph{}
Analizując powyższe przykłady można dojść do wniosku, że złożona kwadratura trapezów nie jest najlepszą metoda obliczania całki z funkcji dwóch zmiennych. Jednak warto zauważyć, że jest ona bardzo prosta w założeniach oraz implementacji, także może posłużyć jako punkt wyjścia dla bardziej skomplikowanych metod.
\end{document}